\documentclass[12pt,a4paper]{article}
\usepackage{ctex}
\usepackage{amsmath,amscd,amsbsy,amssymb,latexsym,url,bm,amsthm}
\usepackage{epsfig,graphicx,subfigure}
\usepackage{enumitem,balance}
\usepackage{wrapfig}
\usepackage{mathrsfs,euscript}
\usepackage[usenames]{xcolor}
\usepackage{hyperref}
\usepackage[vlined,ruled,linesnumbered]{algorithm2e}
\hypersetup{colorlinks=true,linkcolor=black}

\newtheorem{theorem}{Theorem}
\newtheorem{lemma}[theorem]{Lemma}
\newtheorem{proposition}[theorem]{Proposition}
\newtheorem{corollary}[theorem]{Corollary}
\newtheorem{exercise}{Exercise}
\newtheorem*{solution}{Solution}
\newtheorem{definition}{Definition}
\theoremstyle{definition}

\renewcommand{\thefootnote}{\fnsymbol{footnote}}

\newcommand{\postscript}[2]
 {\setlength{\epsfxsize}{#2\hsize}
  \centerline{\epsfbox{#1}}}

\renewcommand{\baselinestretch}{1.0}

\setlength{\oddsidemargin}{-0.365in}
\setlength{\evensidemargin}{-0.365in}
\setlength{\topmargin}{-0.3in}
\setlength{\headheight}{0in}
\setlength{\headsep}{0in}
\setlength{\textheight}{10.1in}
\setlength{\textwidth}{7in}
\makeatletter \renewenvironment{proof}[1][Proof] {\par\pushQED{\qed}\normalfont\topsep6\p@\@plus6\p@\relax\trivlist\item[\hskip\labelsep\bfseries#1\@addpunct{.}]\ignorespaces}{\popQED\endtrivlist\@endpefalse} \makeatother
\makeatletter
\renewenvironment{solution}[1][Solution] {\par\pushQED{\qed}\normalfont\topsep6\p@\@plus6\p@\relax\trivlist\item[\hskip\labelsep\bfseries#1\@addpunct{.}]\ignorespaces}{\popQED\endtrivlist\@endpefalse} \makeatother

\begin{document}
\noindent

%========================================================================
\noindent\framebox[\linewidth]{\shortstack[c]{
\Large{\textbf{CS499 Homework 5}}\vspace{1mm}}}
\begin{center}
% Please write down your name, student id and email.
\footnotesize{\color{blue}$*$ Name:Zehao Wang\quad Student ID:518021910976}
\end{center}

\begin{enumerate}
    \item 
        What is $11^4$? Why is this number easy to compute, for a person who knows binomial coefficients?
        \begin{solution}
        	$11^4=14641$.
        	
        	\begin{equation*}
        	    11^4=(10+1)^4
        	\end{equation*}
            \begin{equation*}    
        	    =\sum_{k} \binom{4}{k} 10^k\times 1^{4-k}
        	\end{equation*}  
            \begin{equation*} 
        	    =\sum_{k=0}^{4} \binom{4}{k} 10^k
        	\end{equation*}
            \begin{equation*}   
        	    =14641
        	\end{equation*}
        \end{solution}
    \item 
        For which value(s) of $k$ is $\binom{n}{k}$ a maximum, when $n$ is a given positive integer? Prove your answer.
        \begin{solution}
        	For any integer $n$, we have:
        	\begin{equation*}
        	    \binom{n}{k}=\frac{n^{\underline{k}}}{k!}
        	\end{equation*}
        	
        	Assume that 
        	\begin{equation}\label{T2_1}
        	   \frac {\binom{n}{k}}{\binom{n}{k-1}}=\frac{n-k+1}{k}\ge 1
        	\end{equation}
        	
        	and
        	
        	\begin{equation}\label{T2_2}
        	   \frac{\binom{n}{k}}{\binom{n}{k+1}}=\frac{k+1}{n-k}\ge 1
        	\end{equation}
        	
        	According to (\ref{T2_1}), we can get $k\le \frac{n+1}{2}$.
        	
        	According to (\ref{T2_2}), we can get $k \ge \frac{n-1}{2}$.
        	
        	Therefore,
        	
        	When $n$ is an odd number and $k=\frac{n+1}{2}$ or $\frac{n-1}{2}$, $\binom{n}{k}$ is a maximum.
        	
        	When $n$ is an even number and $k=\frac{n}{2}$, $\binom{n}{k}$ is a maximum.
        	
        \end{solution}
    \item 
        Prove the hexagon property,
        \begin{equation*}
            \binom{n-1}{k-1}\binom{n}{k+1}\binom{n+1}{k}=\binom{n-1}{k}\binom{n+1}{k+1}\binom{n}{k-1}
        \end{equation*}
        
        \begin{proof}
        	\begin{equation*}
        	    \binom{n-1}{k-1}\binom{n}{k+1}\binom{n+1}{k}
        	\end{equation*}
        	\begin{equation*}
        	    =\frac{(n-1)^{\underline{k-1}}}{(k-1)!}\times\frac{n^{\underline{k+1}}}{(k+1)!}\times\frac{(n+1)^{\underline{k}}}{k!}
        	\end{equation*}
        	\begin{equation*}
        	    =\frac{(n-1)(n-2)\cdots(n-k+1)}{(k-1)!}\times\frac{n(n-1)(n-2)\cdots(n-k)}{(k+1)!}\times\frac{(n+1)n\cdots(n-k+2)}{k!}
        	\end{equation*}
        	\begin{equation*}
        	    =\frac{n(n-1)(n-2)\cdots(n-k)}{(k-1)!}\times\frac{(n-1)(n-2)\cdots(n-k)}{k!}\times\frac{(n+1)n(n-1)\cdots(n-k+2)(n-k+1)}{(k+1)!}
        	\end{equation*}
        	\begin{equation*}
        	    =\frac{n^{\underline{k-1}}}{(k-1)!}\times\frac{(n-1)^{\underline{k}}}{k!}\times\frac{(n+1)^{\underline{k+1}}}{(k+1)!}
        	\end{equation*}
        	\begin{equation*}
        	    =\binom{n-1}{k}\binom{n+1}{k+1}\binom{n}{k-1}
        	\end{equation*}
        	
        \end{proof}
    \item 
        Evaluate $\binom{-1}{k}$ by negating (actually un-negating) its upper index.
        \begin{solution}
        	We have 
        	\begin{equation*}
        	    \binom{n}{k}=(-1)^k\binom{n-k-1}{k!}
        	\end{equation*}
        	
        	Therefore
        	\begin{equation*}
        	    \binom{-1}{k}=(-1)^k\binom{k-(-1)-1}{k}=(-1)^k\binom{k}{k}=(-1)^k[k\ge 0].
        	\end{equation*}
        \end{solution}
    \item 
        Let $p$ be prime. Show that $\binom{p}{k}\ mod\ p=0$ for $0<k<p$. What does this imply about the binomial coefficients $\binom{p-1}{k}$?
        \begin{solution}
        	Because
        	
        	\begin{equation*}
        	\binom{p}{k}=\frac{p^{\underline{k}}}{k!}=p\times\frac{(p-1)^{\underline{k-1}}}{k!}        	\end{equation*}
        	
        Therefore $\binom{p}{k}$ has a factorial $p$. So it is divisible by $p$, namely $\binom{p}{k}\ mod\ p=0$.
        
        Because we have
        \begin{equation*}
            \binom{p}{k}=\binom{p-1}{k-1}+\binom{p-1}{k}
        \end{equation*}
        
        and
        \begin{equation*}
            \binom{p}{k}\equiv 0(mod\ p)
        \end{equation*}
        
        Therefore
        \begin{equation}\label{EQ}
            \binom{p-1}{k-1}+\binom{p-1}{k}\equiv 0(mod\ p)
        \end{equation}
        
        As for $\binom{p-1}{k-1}$,we have
        \begin{equation*}
            \binom{p-1}{k-1}=\frac{(p-1)\cdots(p-k+1)}{(k-1)!}\equiv\frac{(-1)(-2)\cdots(-k+1)}{(k-1)!}(mod\ p)
        \end{equation*}
        Because
        \begin{equation*}
            \frac{(-1)(-2)\cdots(-k+1)}{(k-1)!}=(-1)^{k-1}
        \end{equation*}
        
        Therefore $\binom{p-1}{k-1}\equiv (-1)^{k-1}(\mod p)$.
        
        According to (\ref{EQ}), we can know $\binom{p}{k-1}\equiv (-1)^k(mod\ p)$.
    \end{solution}
    \item 
        Fix up the text's derivation in Problem 6, Section 5.2, by correctly applying symmetry.
        \begin{solution}
        	 \begin{equation*}
        	     \frac{1}{n} \sum_{k} \binom{n+k}{k} \binom{n+1}{k+1}(-1)^k
        	 \end{equation*}
        	 \begin{equation*}
        	     =\frac{1}{n+1} \sum_{k\ge 0} \binom{n+k}{n} \binom{n+1}{k+1}(-1)^k
        	 \end{equation*}
        	 \begin{equation*}
        	     =\frac{1}{n+1} \sum_{k} \binom{n+k}{n} \binom{n+1}{k+1}(-1)^k
        	 \end{equation*}
        	 \begin{equation*}
        	     =-\frac{1}{n+1} \binom{n-1}{n}\binom{n+1}{0}(-1)^{-1}
        	 \end{equation*}
        \end{solution}
    \item 
        Is (5.34) true also when $k<0$?
        \begin{solution}
            Yes.Proof is as follows.
            
            When $k<0$, 
            \begin{equation*}
                r^{\underline{k}}(r-\frac{1}{2})^{\underline{k}}
            \end{equation*}
            \begin{equation*}
                =\frac{1}{(r+1)(r+2)\cdots(r+|k|)}\times \frac{1}{(r+\frac{1}{2})(r+\frac{1}{2}+1)\cdots(n+\frac{1}{2}+|k|-1)}
            \end{equation*}
            \begin{equation*}
                =\frac{2^{|k|}}{(2r+2)(2r+4)\cdots(2r+2|k|)}\times\frac{2^{|k|}}{(2r+1)(2r+3)(2r+5)\cdots(2r+2|k|-1)}
            \end{equation*}
            \begin{equation*}
                =2^{2|k|}\times (2r)^{\underline{2|k|}}
            \end{equation*}
            \begin{equation*}
                =\frac{(2r)^{\underline{2k}}}{2^{2k}}
            \end{equation*}
        \end{solution}
    \item 
        Evaluate 
        \begin{equation*}
            \sum_{k} \binom{n}{k} (-1)^k(1-\frac{k}{n})^n
        \end{equation*}
        What is the approximate value of this sum, when $n$ is very large?
        \begin{solution}
        	According to (5.42):
        	\begin{equation*}
        	   \sum_{k}\binom{n}{k}(-1)^k(a_0+a_1k+\cdots+a_kk^n)=(-1)^nn!a_n
        	\end{equation*}
        	We can derive that:
        	\begin{equation*}
        	    \sum_{k} \binom{n}{k} (-1)^k(1-\frac{k}{n})^n
        	\end{equation*}
        	\begin{equation*}
        	     =(-1)^n\times n!\times (-1)^n\times (\frac{1}{n})^n
        	\end{equation*}
        	\begin{equation*}
        	    =\frac{n!}{n^n}
        	\end{equation*}
        	According to Stirling's Approximation:
        	\begin{equation*}
        	    n!\sim\sqrt{2\pi n}(\frac{n}{e})^n
        	\end{equation*}
        	We can know that when $n$ is very large,
        	\begin{equation*}
        	    \frac{n!}{n^n}\sim\frac{\sqrt{2\pi n}}{e^n}
        	\end{equation*}
        \end{solution}
    \item 
        Show that the generalized exponentials of (5.58) obey the law
        \begin{equation*}
            \epsilon_t(z)=\epsilon(tz)^{\frac{1}{t}},\ if\ t\neq 0
        \end{equation*}
        where $\epsilon(z)$ is an abbreviation for $\epsilon_1(z)$.
        \begin{solution}
        	\begin{equation*}
        	    \epsilon_t(z)^{t}
        	\end{equation*}
        	\begin{equation*}
        	    =\sum_{k\ge 0}t(tk+t)^{k-1}\frac{z^k}{k!}
        	\end{equation*}
        	\begin{equation*}
        	    =\sum_{k\ge 0}t^k(k+1)^{k-1}\frac{z^k}{k!}
        	\end{equation*}
        	\begin{equation*}
        	    =\sum_{k\ge 0}(k+1)^{k-1}\frac{(tz)^k}{k!}
        	\end{equation*}
        	\begin{equation*}
        	    =\epsilon(tz)
        	\end{equation*}
        	Therefore
        	\begin{equation*}
        	    \epsilon_t(z)=\epsilon(tz)^{\frac{1}{t}},\ if\ t\neq 0
        	\end{equation*}
        \end{solution}
\end{enumerate}

%========================================================================
\end{document}
