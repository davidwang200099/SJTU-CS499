\documentclass[12pt,a4paper]{article}
\usepackage{ctex}
\usepackage{amsmath,amscd,amsbsy,amssymb,latexsym,url,bm,amsthm}
\usepackage{epsfig,graphicx,subfigure}
\usepackage{enumitem,balance}
\usepackage{wrapfig}
\usepackage{mathrsfs,euscript}
\usepackage[usenames]{xcolor}
\usepackage{hyperref}
\usepackage[vlined,ruled,linesnumbered]{algorithm2e}
\hypersetup{colorlinks=true,linkcolor=black}

\newtheorem{theorem}{Theorem}
\newtheorem{lemma}[theorem]{Lemma}
\newtheorem{proposition}[theorem]{Proposition}
\newtheorem{corollary}[theorem]{Corollary}
\newtheorem{exercise}{Exercise}
\newtheorem*{solution}{Solution}
\newtheorem{definition}{Definition}
\theoremstyle{definition}

\renewcommand{\thefootnote}{\fnsymbol{footnote}}

\newcommand{\postscript}[2]
 {\setlength{\epsfxsize}{#2\hsize}
  \centerline{\epsfbox{#1}}}

\renewcommand{\baselinestretch}{1.0}

\setlength{\oddsidemargin}{-0.365in}
\setlength{\evensidemargin}{-0.365in}
\setlength{\topmargin}{-0.3in}
\setlength{\headheight}{0in}
\setlength{\headsep}{0in}
\setlength{\textheight}{10.1in}
\setlength{\textwidth}{7in}
\makeatletter \renewenvironment{proof}[1][Proof] {\par\pushQED{\qed}\normalfont\topsep6\p@\@plus6\p@\relax\trivlist\item[\hskip\labelsep\bfseries#1\@addpunct{.}]\ignorespaces}{\popQED\endtrivlist\@endpefalse} \makeatother
\makeatletter
\renewenvironment{solution}[1][Solution] {\par\pushQED{\qed}\normalfont\topsep6\p@\@plus6\p@\relax\trivlist\item[\hskip\labelsep\bfseries#1\@addpunct{.}]\ignorespaces}{\popQED\endtrivlist\@endpefalse} \makeatother

\begin{document}
\noindent

%========================================================================
\noindent\framebox[\linewidth]{\shortstack[c]{
\Large{\textbf{CS499 Homework 6}}\vspace{1mm}}}
\begin{center}
% Please write down your name, student id and email.
\footnotesize{\color{blue}$*$ Name:Zehao Wang\quad Student ID:518021910976}
\end{center}

\begin{enumerate}
    \item 
        What are the ${a \brack b}=11$ premutations of {1,2,3,4} that have exactly two cycles?
        \begin{solution}
        There are 11 different ways to make two cycles from four elements, and the coordinating permutations are as follows.
        	
        	    [1,2,3][4]:2314
        	 
        	    [1,2,4][3]:2431
        	 
        	    [1,3,4][2]:3241
        	 
        	    [2,3,4][1]:1342
        	 
        	    [1,3,2][4]:3124
        
        	    [1,4,2][3]:4132
        	 
        	    [1,4,3][2]:4213
        	 
        	    [2,4,3][1]:1423
        	 
        	    [1,2][3,4]:2143
        	 
        	    [1,3][2,4]:3412
        	 
        	    [1,4][2,3]:4321
        
        \end{solution}
    \item  
       There are $m^n$ functions from a set of $n$ elements into a set of $m$ elements. How many of them range over exactly $k$ different values?
       \begin{solution}
       	    To determine the number of functions ranging over exactly $k$ different values, we can first partition the $n$ elements into $k$ groups and assign an element from the $m$ elements to each of the $k$ groups.
       	    
       	    By definition, there are $ n \brace k$ ways to partition the $n$ elements into $k$ groups.
       	    
       	    To assign an element from $m$ elements to each of the $k$ groups, we have $m^{\underline{k}}$ ways.
       	    
       	    By theorem of multiplication, $m^{\underline{k}}{n \brace k}$ of the $m^n$ elements range over exactly $k$ different values.
       \end{solution}
    \item 
        Card stackers in the real world know that it's wise to allow a bit of slack so that the cards will not topple over when a breath of wind comes along. Suppose the center of gravity of the top $k$ cards is required to be at least $\epsilon$ units from the edge of the $(k+1)$-st card.
        \begin{solution}
        	This time we have 
        	\begin{equation*}
        	    d_{k+1}=\frac{(d_1+1-\epsilon)+(d_2+1-\epsilon)+\cdots+(d_k+1-\epsilon)}{k} 
        	\end{equation*}
        	By solving this recurrence we can get
        	\begin{equation*}
        	    d_{k+1}=(1-\epsilon)H_k
        	\end{equation*}
        	
        	As long as $\epsilon<1$, it is unbounded.
        	
        \end{solution}
    %\item 
        %Express $1+\frac{1}{3}+\cdots+\frac{1}{2n+1}$ in terms of harmonic numbers.
        %\begin{solution}
        %	\begin{equation*}
        %	    1+\frac{1}{3}+\cdots+\frac{1}{2n+1}
        %	\end{equation*}
        %	\begin{equation*}
        %	    =\sum_{i=1}^{2n+2} \frac{1}{i}-(\frac{1}{2}+\frac{1}{4}+\cdots+\frac{1}{2n+2})
        %	\end{equation*}
        %	\begin{equation*}
        %	    =\sum_{i=1}^{2n+2}\frac{1}{i}-\frac{1}{2}\sum_{i=1}^{n+1} \frac{1}{i}
        %	\end{equation*}
        %	\begin{equation*}
        %	    =H_{2n+2}-\frac{1}{2}H_{n+1}.
        %	\end{equation*}
        %\end{solution}
    %\item 
     %   Explain how to get the recurrence (6.75) from the definition of $U_n(x,y)$ in (6.74), and solve the %recurrence.
    %\item 
     %   
    \item 
        An explorer has left a pair of baby rabbits n an island. If baby rabbits become adults after one month, and if each pair of adult rabbits produces one pair of baby rabbits every month, how many pairs of rabbits are present after $n$ months?
        \begin{solution}
        	The number of pairs of baby rabbits forms a Fibonacci sequence.
        	
        	There are $F_{n+1}$ pairs of rabbits, $F_n$ pairs of which are adults and $F_{n-1}$ pairs of which are babies.
        	
        \end{solution}
    \item 
        Show that Cassini's identity (6.103) is a special case of (6.108), and a special case of (6.134)
        \begin{solution}
        	\ \\
        	\begin{enumerate}
        		\item 
        		    By definition, we have
        		    \begin{equation}\label{E2}
        		    F_{-n}=(-1)^nF_n
        		    \end{equation}
        		    and (6.103) is:
        		    \begin{equation}\label{E3}
        		    F_{n+1}F_{n-1}-F_n^2=(-1)^n
        		    \end{equation}
        		    
        		    Let $k=1-n$, According to (6.108), we have
        		    \begin{equation*}
        		    F_1=(-1)^{n-1}F_{n-1}F_{n+1}+(-1)^nF_n^2
        		    \end{equation*}
        		    
        		    By definition, $F_1=1$.
        		    
        		    Therefore 
        		    \begin{equation*}
        		    F_{n-1}F_{n+1}-F_n^2=(-1)^n
        		    \end{equation*}
        		    which is (6.103).
        		\item 
        		    In (6.134), let $m=1$, $k=n-1$ and $x_i=1$, we have
        		    \begin{equation*}
        		        K_{n+1}(1,1,\cdots,1)K_{n-1}(1,1,\cdots,1)=K_n(1,1,\cdots,1)K_n(1,1,\cdots,1)+(-1)^{n-1}K_0K_0
        		    \end{equation*}
        		    
        		    By definition,$K_n(1,1,1,\cdots,1)=F_n$.
        		    
        		    Therefore we have
        		    \begin{equation*}
        		        F_{n+1}F_{n-1}=F_n^2+(-1)^n
        		    \end{equation*}
        		    
        		    which is (6.103).
        	\end{enumerate}
        	
        	
        \end{solution}
    \item 
        Use the Fibonacci number system to convert 65 mi/hr into an approximate number of km/hr.
        \begin{solution}
        	Use greedy approach to transform 65 into its Fibonacci form $55+8+2$ and let each of them turn into the next number in the Fibonacci sequence and we can get $89+13+3=105$.
        \end{solution}
    \item 
        What is $\sum_{k}(-1)^k {n \brack k}$, the row sum of Stirling's cycle-number triangle with alternating signs, when $n$ is a nonnegative integer?
       \begin{solution}
    	We have the formula
    	\begin{equation}\label{E1}
    	    x^{\overline{n}}=\sum_{k}{n \brack k}x^k
    	\end{equation}
    
    	Let $x=-1$, the right hand side of (\ref{E1}) is $\sum_{k}(-1)^k$.
    	
    	And the left hand side is $(-1)^{\overline{n}}$.
    	
    	According to the definition of raising power, 
    	\begin{enumerate}
    		\item 
    		    when $n=0$, $(-1)^{\overline{0}}=1$.
    		\item 
    		    when $n=1$, $(-1)^{\overline{1}}=-1$.
    		\item 
    		    when $n\ge 1$, $(-1)^{\overline{n}}=0$.
    		\item 
    		    when $n<0$, by definition $(-1)^{\overline{n}}=0$
    	\end{enumerate}
    	Therefore, the result can be transformed into:
       \begin{equation*}
            \sum_k(-1)^k {n \brack k}=[n=0]-[n=1].
         \end{equation*}
      \end{solution}
    \item 
        Prove that Stirling numbers have an inversion law analogous to (5.48):
        \begin{equation*}
           g(n)=\sum_k{n \brace k}(-1)^kf(k) \Leftrightarrow f(n)=\sum_k{n \brack k}(-1)^kg(k)
        \end{equation*}
        \begin{solution}
        	Assume that the left hand side makes sense.
        	
        	Then for the right hand side,
        	\begin{equation*}
        	    \sum_k{n \brack k}(-1)^kg(k)
        	\end{equation*}
        	\begin{equation*}
        	    =\sum_k{n \brack k}(-1)^k (\sum_j{n \brace j} (-1)^j f(j))
        	\end{equation*}
        	\begin{equation*}
        	    =\sum_k\sum_j{ n \brack k}{k \brace j} (-1)^k (-1)^j f(j)
        	\end{equation*}
        	\begin{equation*}
        	    =\sum_j f(j) (-1)^j \sum_k {n \brack k}{k \brace j}(-1)^k 1^j
        	\end{equation*}
        	\begin{equation*}
        	    =\sum_j f(j) (-1)^{n-j} \sum_k {n \brack k}{k \brace j}(-1)^{n-k} 1^j
        	\end{equation*}
        	\begin{equation*}
        	    =\sum_j f(j) (-1)^{n-j} [n=j]
        	\end{equation*}
        	\begin{equation*}
        	    =f(n).
        	\end{equation*}
        	
        	Assume that the right hand side makes sense, then just the same as above,
        	\begin{equation*}
        	    \sum_k{n \brace k}(-1)^kf(k)
        	\end{equation*}
        	\begin{equation*}
        	    =\sum_k{n \brace k}(-1)^k \sum_j {n \brack j}(-1)^j g(j)
        	\end{equation*}
        	\begin{equation*}
        	    =\sum_j g(j)(-1)^{n-j}[n=j]
        	\end{equation*}
        	\begin{equation*}
        	    =g(n)
        	\end{equation*}
        	
        	Therefore the statement above makes sense.
        \end{solution}
    \item 
        The differential operators $D=\frac{d}{dz}$ and $\vartheta=zD$ are mentioned in Chapters 2 and 5. We have
        \begin{equation*}
            \vartheta^2=z^2D^2+zD
        \end{equation*}
        Prove the general formulas
        \begin{equation*}
            \vartheta^n=\sum_k {n\brace k} z^kD^k
        \end{equation*}
        \begin{equation*}
            z^nD^n=\sum_k{ n \brack k} (-1)^{n-k} \vartheta^k
        \end{equation*}
        \begin{solution}
        	For $f(z)=z^x$, $z^nD^nf(z)=x^{\underline{n}}z^{x}$ and $\vartheta^nf(z)=x^{n}z^x$.
        	
        	Because $\sum_k {n \brace k} (-1)^{n-k} x^{\underline{k}}=x^n$,
        	
        	therefore $\sum_k {n \brace k} (-1)^{n-k} x^{\underline{k}} z^x=x^n z^x$.
        	
        	Therefore $\vartheta^n=\sum_k {n\brace k} z^kD^k$.
        	
        	Because we also have 
        	\begin{equation*}
        	    \sum_k {n \brack k}(-1)^{n-k} x^k=x^{\underline{n}}
        	\end{equation*}
        	
        	So we also have
        	\begin{equation*}
        	    z^nD^n=\sum_k{ n \brack k} (-1)^{n-k} \vartheta^k
        	\end{equation*}
        \end{solution}
   
\end{enumerate}

%========================================================================
\end{document}
