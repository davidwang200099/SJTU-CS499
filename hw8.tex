\documentclass[12pt,a4paper]{ctexart}
\usepackage{ctex}
\usepackage{amsmath,amscd,amsbsy,amssymb,latexsym,url,bm,amsthm}
\usepackage{epsfig,graphicx,subfigure}
\usepackage{enumitem,balance}
\usepackage{wrapfig}
\usepackage{mathrsfs,euscript}
\usepackage[usenames]{xcolor}
\usepackage{hyperref}
\usepackage[vlined,ruled,linesnumbered]{algorithm2e}
\usepackage{listings}
\hypersetup{colorlinks=true,linkcolor=black}

\newtheorem{theorem}{Theorem}
\newtheorem{lemma}[theorem]{Lemma}
\newtheorem{proposition}[theorem]{Proposition}
\newtheorem{corollary}[theorem]{Corollary}
\newtheorem{exercise}{Exercise}
\newtheorem*{solution}{Solution}
\newtheorem{definition}{Definition}
\theoremstyle{definition}

\renewcommand{\thefootnote}{\fnsymbol{footnote}}

\newcommand{\postscript}[2]
 {\setlength{\epsfxsize}{#2\hsize}
  \centerline{\epsfbox{#1}}}

\renewcommand{\baselinestretch}{1.0}
\def\bangle{ \atopwithdelims \langle \rangle}
\setlength{\oddsidemargin}{-0.365in}
\setlength{\evensidemargin}{-0.365in}
\setlength{\topmargin}{-0.3in}
\setlength{\headheight}{0in}
\setlength{\headsep}{0in}
\setlength{\textheight}{10.1in}
\setlength{\textwidth}{7in}
\makeatletter \renewenvironment{proof}[1][Proof] {\par\pushQED{\qed}\normalfont\topsep6\p@\@plus6\p@\relax\trivlist\item[\hskip\labelsep\bfseries#1\@addpunct{.}]\ignorespaces}{\popQED\endtrivlist\@endpefalse} \makeatother
\makeatletter
\renewenvironment{solution}[1][Solution] {\par\pushQED{\qed}\normalfont\topsep6\p@\@plus6\p@\relax\trivlist\item[\hskip\labelsep\bfseries#1\@addpunct{.}]\ignorespaces}{\popQED\endtrivlist\@endpefalse} \makeatother

\begin{document}
\noindent

%========================================================================
\noindent\framebox[\linewidth]{\shortstack[c]{
\Large{\textbf{CS499 Homework 8}}\vspace{1mm}}}
\begin{center}
% Please write down your name, student id and email.
\footnotesize{\color{blue}$*$ Name:Zehao Wang\quad Student ID:518021910976}
\end{center}

\begin{enumerate}
    \item 
        An eccentric collector of $2 \times n$ domino tilings pays \$4 for each vertical domino and \$1 for each horizontal domino. How many tilings are worth exactly \$m by this criterion?
        \begin{solution}
        	We can replace horizontal tilings with $z$ and vertical tilings with $z^4$.
        	
        	Then the generating function will turn into $\frac{1}{1-z^4-z^2}$.
        	
        	Because
        	\begin{equation*}
        	    \frac{1}{1-z-z^2}=\sum_{n \ge 1}F_nz^{n-1}
        	\end{equation*}
        	Therefore 
        	\begin{equation*}
        	    \frac{1}{1-z^2-z^4}=\sum_{n \ge 1}F_nz^{2n-2}
        	\end{equation*}
        	
        	The number of ways to pay for tilings is just the coefficient of $z^m$ entity.
        	
        	Therefore
        	\begin{itemize}
        		\item 
        		    When $m$ is odd, it should be zero.
        		\item 
        		    When $m$ is even, it should be $F_{\frac{m}{2}+1}$.
        	\end{itemize}
        \end{solution}
    \item 
        Give the generating function and the exponential generating function for the sequence $<2,5,13,35,\cdots>$=$2^n+3^n$ in closed form.
        \begin{solution}
        	Generating function in closed form:
        	\begin{equation*}
        	    \sum_{n \ge 0}(2^n+3^n) z^n
        	\end{equation*}
        	\begin{equation*}
        	    =\sum_{n \ge 0} 2^nz^n+3^nz^n
        	\end{equation*}
        	\begin{equation*}
        	    =\sum_{n \ge 0} (2z)^n +(3z)^n
        	\end{equation*}
        	\begin{equation*}
        	    =\frac{1}{1-2z}+\frac{1}{1-3z}
        	\end{equation*}
        	
        	Exponential generating function in closed form:
        	\begin{equation*}
        	    \sum_{n \ge 0} \frac{2^n+3^n}{n!} z^n
        	\end{equation*}
        	\begin{equation*}
        	    =\sum_{n \ge 0} \frac{1}{n!} ((2z)^n+(3z)^n)
        	\end{equation*}
        	\begin{equation*}
        	    =\sum_{n \ge 0} \frac{1}{n!}(2z)^n+\sum_{n \ge 0} \frac{1}{n!}(3z)^n
        	\end{equation*}
        	\begin{equation*}
        	    =e^{2z}+e^{3z}
        	\end{equation*}
        \end{solution}
    \item 
        What is $\sum_{n\ge0} \frac{H_n}{10^n}$?
        \begin{solution}
        	The closed form of generating function of $<H_n>$ is $G(z)=\frac{1}{1-z}\ln \frac{1}{1-z}$.
        	
        	Because the convergence radius $\frac{1}{R}=lim_{n \rightarrow \inf} \sqrt[n]{H_n}=1$, therefore the convergence radius of this function is $1$. Therefore when $z=\frac{1}{10}$, the function is convergent.
        	
        	Therefore
        	\begin{equation*} 
        	\sum_{n\ge0} \frac{H_n}{10^n}=\sum_{n \ge 0} \frac{H_n}{z^n}|_{z=\frac{1}{10}}=\frac{10}{9}\ln \frac{10}{9}
            \end{equation*}
        \end{solution}
    \item 
        Solve the recurrence
        \begin{equation*}
            g_0=1;
        \end{equation*}
        \begin{equation*}
            g_n=g_{n-1}+2g_{n-2}+\cdots+ng_0,\quad for\ n>0
        \end{equation*}
        \begin{solution}
            $<g_n>$ is the convolution of $<n>$ and $<g_n>$. 
            
            According to the rule of convolution, we can know that 
            \begin{equation*}
                G(z)=\frac{z}{(1-z)^2}G(z)+1
            \end{equation*}
            
            Therefore 
            \begin{equation*}
                G(z)=1+\frac{z}{1-3z+z^2}
            \end{equation*}
            
            Therefore $g_n=F_2n+[n=0]$.
        \end{solution}
    \item 
        What is $[z^n](\ln (1-z))^2/(1-z)^{m+1}$?
        \begin{solution}
        	Differentiate $(1-z)^{-x-1}$ twice with respect to $x$, we can obtain that
        	\begin{equation}\label{E1}
        	    \frac{\ln^2(1-z)}{(1-z)^{x+1}}=\sum_{n \ge 0}\binom{x+n}{n}((H_{x+n}-H_x)^2-(H_{x+n}^{(2)}-H_x^{(2)}))z^n.
        	\end{equation}
        	Therefore 
        	\begin{equation*}
        	    [z^n](\ln (1-z))^2/(1-z)^{m+1}=\binom{m+n}{n}((H_{m+n}-H_m)^2-\sum_{i=1}^{n}\frac{2}{(x+i)^3})
        	\end{equation*}
        	
        \end{solution}
    \item 
        previous exercise to evaluate $\sum_{k=0}^{n}H_kH_{n-k}$.
        \begin{solution}
            According to (7.57):
            \begin{equation*}
                \frac{1}{1-z}\ln \frac{1}{1-z}=\sum_n H_nz^n
            \end{equation*}
            we have:
            \begin{equation*}
                \frac{\ln^2 (1-z)}{(1-z)^2}=\sum_n (\sum_k H_kH_{n-k})z^n
            \end{equation*}
            Let $x=1$ in (\ref{E1}), we have:
            \begin{equation*}
                \frac{\ln^2(1-z)}{(1-z)^{2}}=\sum_{n \ge 0}\binom{1+n}{n}((H_{1+n}-H_1)^2-(H_{1+n}^{(2)}-H_1^{(2)}))z^n
            \end{equation*}
            Therefore
            \begin{equation*}
                \sum_{k=0}^{n}H_kH_{n-k}
            \end{equation*}
            \begin{equation*}
                =\binom{1+n}{n}((H_{1+n}-H_1)^2-(H_{1+n}^{(2)}-H_1^{(2)}))
            \end{equation*}
            \begin{equation*}
                =(n+1)(H_n^2-H_n^{(2)})-2n(H_n-1)
            \end{equation*}
        \end{solution}
    \item 
        A robber holds up a bank and demands \$500 in tens and twenties. He also demands to know the number of ways in which the cashier can give him the money. Find a generating function $G(z)$ for which this number is $[z^{500}]G(z)$, and a more compact generating function $\check{G}(z)$ for which this number is $[z^{50}]\check{G}(z)$. Determine the required number of ways by (a).using partial fractions; (b) using a method like (7.39).
        \begin{solution}
        	This is giving change with demonimations $10$ and $20$. Therefore 
        	\begin{equation*}
        	    G(z)=\frac{1}{(1-z^10)(1-z^20)}=\check{G}(z^{10})
        	\end{equation*}
        	\begin{enumerate}
        		\item 
        		    The partial fraction decomposition of $\check{G}(z)$ is:
        		    \begin{equation*}
        		        \check{G}(z)=\frac{1}{2}\frac{1}{(1-z)^2}+\frac{1}{4}\frac{1}{1-z}+\frac{1}{4}\frac{1}{1+z}
        		    \end{equation*}
        		    
        		    Therefore
        		    \begin{equation*}
        		        [z^n]\check{G}(z)=\frac{2n+3+(-1)^n}{4}
        		    \end{equation*}
        		    
        		    Let $n=50$ and we can get $26$ ways to make the payment.
        		\item 
        		    Because
        		    \begin{equation*}
        		        \check{G}(z)=\frac{1}{(1-z)(1-z^2)}=\frac{1+z}{(1-z^2)^2}=(1+z)(1+2z^2+3z^4+\cdots)
        		    \end{equation*}
        		    
        		    Therefore
        		    \begin{equation*}
        		        [z^n]G(z)=\lfloor \frac{n}{2} \rfloor +1
        		    \end{equation*}
        		    
        		    Let $n=50$ and we can get 26 ways.
        	\end{enumerate}
        \end{solution}
\end{enumerate}

%========================================================================
\end{document}
