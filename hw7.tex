\documentclass[12pt,a4paper]{article}
\usepackage{ctex}
\usepackage{amsmath,amscd,amsbsy,amssymb,latexsym,url,bm,amsthm}
\usepackage{epsfig,graphicx,subfigure}
\usepackage{enumitem,balance}
\usepackage{wrapfig}
\usepackage{mathrsfs,euscript}
\usepackage[usenames]{xcolor}
\usepackage{hyperref}
\usepackage[vlined,ruled,linesnumbered]{algorithm2e}
\hypersetup{colorlinks=true,linkcolor=black}

\newtheorem{theorem}{Theorem}
\newtheorem{lemma}[theorem]{Lemma}
\newtheorem{proposition}[theorem]{Proposition}
\newtheorem{corollary}[theorem]{Corollary}
\newtheorem{exercise}{Exercise}
\newtheorem*{solution}{Solution}
\newtheorem{definition}{Definition}
\theoremstyle{definition}

\renewcommand{\thefootnote}{\fnsymbol{footnote}}

\newcommand{\postscript}[2]
 {\setlength{\epsfxsize}{#2\hsize}
  \centerline{\epsfbox{#1}}}

\renewcommand{\baselinestretch}{1.0}
\def\bangle{ \atopwithdelims \langle \rangle}
\setlength{\oddsidemargin}{-0.365in}
\setlength{\evensidemargin}{-0.365in}
\setlength{\topmargin}{-0.3in}
\setlength{\headheight}{0in}
\setlength{\headsep}{0in}
\setlength{\textheight}{10.1in}
\setlength{\textwidth}{7in}
\makeatletter \renewenvironment{proof}[1][Proof] {\par\pushQED{\qed}\normalfont\topsep6\p@\@plus6\p@\relax\trivlist\item[\hskip\labelsep\bfseries#1\@addpunct{.}]\ignorespaces}{\popQED\endtrivlist\@endpefalse} \makeatother
\makeatletter
\renewenvironment{solution}[1][Solution] {\par\pushQED{\qed}\normalfont\topsep6\p@\@plus6\p@\relax\trivlist\item[\hskip\labelsep\bfseries#1\@addpunct{.}]\ignorespaces}{\popQED\endtrivlist\@endpefalse} \makeatother

\begin{document}
\noindent

%========================================================================
\noindent\framebox[\linewidth]{\shortstack[c]{
\Large{\textbf{CS499 Homework 7}}\vspace{1mm}}}
\begin{center}
% Please write down your name, student id and email.
\footnotesize{\color{blue}$*$ Name:Zehao Wang\quad Student ID:518021910976}
\end{center}

\begin{enumerate}
    \item 
        Express $1+\frac{1}{3}+\cdots+\frac{1}{2n+1}$ in terms of harmonic numbers.
        \begin{solution}
        	\begin{equation*}
        	    1+\frac{1}{3}+\cdots+\frac{1}{2n+1}
        	\end{equation*}
        	\begin{equation*}
        	    =\sum_{i=1}^{2n+1} \frac{1}{i}-(\frac{1}{2}+\frac{1}{4}+\cdots+\frac{1}{2n})
        	\end{equation*}
        	\begin{equation*}
        	    =\sum_{i=1}^{2n+1}\frac{1}{i}-\frac{1}{2}\sum_{i=1}^{n} \frac{1}{i}
        	\end{equation*}
        	\begin{equation*}
        	    =H_{2n+1}-\frac{1}{2}H_{n}.
        	\end{equation*}
        \end{solution}
    \item 
        Explain how to get the recurrence (6.75) from the definition of $U_n(x,y)$ in (6.74), and solve the recurrence.
        \begin{solution}
        	$U_n(x,y)=\sum_{k \ge 1}\binom{n}{k} \frac{(-1)^{k-1}}{k} (x+ky)^n$
        	
        	$=\sum_{k \ge 1}(\binom{n-1}{k}+\binom{n-1}{k-1})(x+ky)\times\frac{(-1)^{k-1}}{k}\times(x+ky)^{n-1}$
        	
        	$=xU_{n-1}(x,y)+\sum_{k \ge 1}(x\binom{n-1}{k-1}+ky\binom{n}{k})\times\frac{(-1)^{k-1}}{k}(x+ky)^{n-1}$
        	
        	$=xU_{n-1}(x,y)+\sum_{k \ge 1}(\frac{kx}{n}\binom{n}{k}+ky\binom{n}{k})\times\frac{(-1)^{k-1}}{k}(x+ky)^{n-1}$
        	
        	$=xU_{n-1}(x,y)+(y+\frac{x}{n})\sum_{k \ge 1}\binom{n}{k}(-1)^{k-1}(x+ky)^{n-1}$
        	
        	$=xU_{n-1}(x,y)+(y+\frac{x}{n})(x^{n-1}+\sum_{k \ge 0}\binom{n}{k}(-1)^{k-1}(x+ky)^{n-1})$
        	
        	Because the $m$-th difference of an (m-1)-degree polynomial is 0,therefore
        	
            \begin{equation*}
                \Delta^n(\frac{x}{y})^{n-1}=\sum_{k} \binom{n}{k} (-1)^{n-k} (\frac{x}{y}+k)^{n-1}=0
            \end{equation*}
            
            Therefore 
            \begin{equation*}
                U_n(x,y)=xU_{n-1}(x,y)+x^n+yx^{n-1}
            \end{equation*}
            
            Then we have
            \begin{equation*}
                \frac{U_n(x,y)}{x^n}=\frac{U_{n-1}(x,y)}{x^{n-1}}+\frac{1}{n}+\frac{y}{x} 
            \end{equation*}
            \begin{equation*}
                =U_0(x,y)+H_n+\frac{ny}{x} 
            \end{equation*}
            
            Therefore 
            \begin{equation*}
                U_n(x,y)=x^nH_n+nx^{n-1}y
            \end{equation*}
        \end{solution}
    \item 
        About how many square kilometers are in 8 square miles?
        \begin{solution}
        	When the unit is square kilometer rather than kilometer, we can shift the Fibonacci numbers twice.
        	
        	The Fibonacci sequence is 1,1,2,3,5,8,13,21,...
        	
        	Therefore there are 21 square kilometers in 8 square miles
        \end{solution}
    \item 
        What is the continued fraction representation of $\Phi$?
        \begin{solution}
        	$a_0,a_1,\cdots,$ all equals 1. Because we have the identity $\Phi=1+\frac{1}{\Phi}$.
        \end{solution}
    \item 
        Prove the power identity (6.37) for Eulerian numbers.
        \begin{solution}
        	\ \\
        	\begin{itemize}
        		\item 
        	    	When $n=0$, $\sum_{k}{n \bangle k}\binom{x+k}{n}={0 \bangle 0}\binom{x}{0}=1=x^0$. The identity is true.
        	    \item 
        	        Assumes that when $n=m-1$, the statement is true. Then when $n=m$, we have:
        	        \begin{equation}\label{E1}
        	            x^m=x^{m-1}\times x=\sum_{k}{m-1 \bangle k}\binom{x+k}{m-1} \times x
        	        \end{equation}
        	        
        	        We have the identity:
        	        \begin{equation}\label{E2}
        	            x=\frac{(k+1)(x+k-m+1)}{m}+\frac{(m-k-1)(x+k+1)}{m} 
        	        \end{equation}
        	        
        	        Putting (\ref{E2}) into (\ref{E1}), we have:
        	        \begin{equation}
        	            x^m=\sum_{k} {m-1 \bangle k}\binom{x+k}{m-1}(\frac{(k+1)(x+k-m+1)}{m}+\frac{(m-k-1)(x+k+1)}{m})
        	        \end{equation}
        	        
        	        Because 
        	        \begin{equation*}
        	            \sum_k {m-1 \bangle k}\binom{x+k}{m-1}\frac{(k+1)(x+k-m+1)}{m}
        	        \end{equation*}
        	        \begin{equation*}
        	            =\sum_{k} (k+1){m-1 \bangle k}(\binom{x+k}{m-1}\times\frac{x+k-m+1}{m})
        	        \end{equation*}
        	        \begin{equation*}
        	            =\sum_{k} (k+1){m-1 \bangle k}\binom{x+k}{m}
        	        \end{equation*}
        	        and
        	        \begin{equation*}
        	            \sum_k {m-1 \bangle k}\binom{x+k}{m-1}\frac{(m-k-1)(x+k+1)}{m}
        	        \end{equation*}
        	        \begin{equation*}
        	            =\sum_k (m-k-1) {m-1 \bangle k}(\binom{x+k}{m-1}\times\frac{x+k+1}{m})
        	        \end{equation*}
        	        \begin{equation*}
        	            =\sum_{j} (m-j){m-1 \bangle j-1}\binom{x+j}{m}
        	        \end{equation*}
        	        \begin{equation*}
        	            =\sum_k (m-k){m-1 \bangle k-1}\binom{x+k}{m}
        	        \end{equation*}
        	        Therefore
        	        \begin{equation*}
        	            \sum_{k} {m-1 \bangle k}\binom{x+k}{m-1}\frac{(k+1)(x+k-m+1)}{m}+\sum_{k}{m-1 \bangle k}\binom{x+k}{m-1}\frac{(m-k-1)(x+k+1)}{m}
        	        \end{equation*}
        	        \begin{equation*}
        	            =\sum_k ((k+1){m-1 \bangle k}+(m-k){m-1 \bangle k-1})\binom{x+k}{m}
        	        \end{equation*}
        	        \begin{equation*}
        	            =\sum_k {m \bangle k}\binom{x+k}{m}
        	        \end{equation*}
        	        Then we prove that
        	        \begin{equation*}
        	            x^m=\sum_k{m \bangle k}\binom{x+k}{m}
        	        \end{equation*}
        	    \item 
        	        Therefore, by induction, we can prove the Worpitzky's identity. 
        	\end{itemize}
        \end{solution}
    \item 
        Prove the Eulerian identity (6.39) by taking the $m$-th difference of (6.37).
        \begin{solution}
        	Because
        	\begin{equation*}
        	    \Delta(\binom{x+k}{n})=\binom{x+k}{n-1}
        	\end{equation*}
        	Therefore
        	\begin{equation*}
        	    \Delta^m(\binom{x+k}{n})=\binom{x+k}{n-m}
        	\end{equation*}
        	Because
        	\begin{equation*}
        	    x^n=\sum_k {n \bangle k}\binom{x+k}{n}
        	\end{equation*}
        	Therefore
        	\begin{equation*}
        	    \Delta^m(x^n)=\sum_{k} {n \bangle k}\binom{x+k}{n}
        	\end{equation*}
        	When $x=0$, we have
        	\begin{equation*}
        	    \binom{k}{n-m}=\sum_{k}{n \bangle k}\binom{k}{n}
        	\end{equation*}
        	
        	According to (6.19):
        	\begin{equation*}
        	    m!{n \brace m}=\sum_k \binom{m}{k}k^n(-1)^{m-k}
        	\end{equation*}
        	
        	We also have
        	\begin{equation*}
        	    \Delta^m (x^n)=\sum_{k} \binom{n}{k} (-1)^{n-k} (x+k)^n
        	\end{equation*}
        	
        	Therefore 
        	\begin{equation*}
        	    \sum_k \binom{n}{k} (-1)^{n-k} (x+k)^n=\sum_{k} {n \bangle k}\binom{x+k}{n}
        	\end{equation*}
        	Therefore
        	\begin{equation*}
        	    \sum_k \binom{n}{k} (-1)^{n-k} k^n=\sum_{k} {n \bangle k}\binom{k}{n}
        	\end{equation*}
        	Therefore
        	\begin{equation*}
        	    m!{n \brace m}=\sum_k {n \bangle k}\binom{k}{n-m}
        	\end{equation*}
        \end{solution}
    \item 
        What is the general solution of the double recurrence
        \begin{equation*}
            A_{n,0}=a_n[n \ge 0];\quad A_{0,k}=0,\quad if\ k>0;
        \end{equation*}
        \begin{equation*}
            A_{n,k}=kA_{n-1,k}+A_{n-1,k-1},\quad integers\ k,n,
        \end{equation*}
        when $k$ and $n$ range over the set of all integers?
        \begin{solution}
        	The recursion $A_{n,k}=kA_{n-1,k}+A_{n-1,k-1}$ looks like that of Stirling number of the second kind, but the values $A_{n,0}$ are different.
        	
        	I guess the answer should be
        	\begin{equation*}
        	    A_{n,k}=\sum_{j\ge 0} a_j {n-j \brace k}
        	\end{equation*}
        	
        	Then we can Prove it by induction.
        	
        	\begin{itemize}
        		\item
        		    When $n=0$ and $k\neq 0$, it is obvious that $A_{0,k}=a_0{0 \brace k}=0$.
        		\item 
        		    When $n\neq 0$ and $k=0$, $A_{n,0}=\sum_{j \ge 0}a_j{n-j \brace 0}=a_n$.
        		\item 
        		    When $n=0$ and $k=0$, $A_{0,0}=a_0$.
        		\item 
        		    Then when $n>0$ and $k>0$, assume that $A_{n-1,k}=\sum_{j\ge 0} a_j {n-1-j \brace k}$ and $A_{n-1,k-1} =\sum_{j\ge 0} a_j {n-1-j \brace k-1}$
        		    
        		    Then 
        		    \begin{equation*}
        		        A_{n,k}=k\times \sum_{j\ge 0} a_j {n-1-j \brace k}+\sum_{j\ge 0} a_j {n-1-j \brace k-1}
        		    \end{equation*}
        		    \begin{equation*}
        		        =\sum_{j \ge 0} a_j(k{n-1-j \brace k}{n-1-j \brace k-1})
        		    \end{equation*}
        		    \begin{equation*}
        		        =\sum_{j\ge 0} a_j{n-j \brace k}
        		    \end{equation*}
        		\item 
        		    Therefore, the statement is true.
        	\end{itemize}
        \end{solution}
   
\end{enumerate}

%========================================================================
\end{document}
