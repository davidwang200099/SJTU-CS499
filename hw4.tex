\documentclass[12pt,a4paper]{article}
\usepackage{ctex}
\usepackage{amsmath,amscd,amsbsy,amssymb,latexsym,url,bm,amsthm}
\usepackage{epsfig,graphicx,subfigure}
\usepackage{enumitem,balance}
\usepackage{wrapfig}
\usepackage{mathrsfs,euscript}
\usepackage[usenames]{xcolor}
\usepackage{hyperref}
\usepackage[vlined,ruled,linesnumbered]{algorithm2e}
\hypersetup{colorlinks=true,linkcolor=black}

\newtheorem{theorem}{Theorem}
\newtheorem{lemma}[theorem]{Lemma}
\newtheorem{proposition}[theorem]{Proposition}
\newtheorem{corollary}[theorem]{Corollary}
\newtheorem{exercise}{Exercise}
\newtheorem*{solution}{Solution}
\newtheorem{definition}{Definition}
\theoremstyle{definition}

\renewcommand{\thefootnote}{\fnsymbol{footnote}}

\newcommand{\postscript}[2]
 {\setlength{\epsfxsize}{#2\hsize}
  \centerline{\epsfbox{#1}}}

\renewcommand{\baselinestretch}{1.0}

\setlength{\oddsidemargin}{-0.365in}
\setlength{\evensidemargin}{-0.365in}
\setlength{\topmargin}{-0.3in}
\setlength{\headheight}{0in}
\setlength{\headsep}{0in}
\setlength{\textheight}{10.1in}
\setlength{\textwidth}{7in}
\makeatletter \renewenvironment{proof}[1][Proof] {\par\pushQED{\qed}\normalfont\topsep6\p@\@plus6\p@\relax\trivlist\item[\hskip\labelsep\bfseries#1\@addpunct{.}]\ignorespaces}{\popQED\endtrivlist\@endpefalse} \makeatother
\makeatletter
\renewenvironment{solution}[1][Solution] {\par\pushQED{\qed}\normalfont\topsep6\p@\@plus6\p@\relax\trivlist\item[\hskip\labelsep\bfseries#1\@addpunct{.}]\ignorespaces}{\popQED\endtrivlist\@endpefalse} \makeatother

\begin{document}
\noindent

%========================================================================
\noindent\framebox[\linewidth]{\shortstack[c]{
\Large{\textbf{CS499 Homework 4}}\vspace{1mm}}}
\begin{center}
% Please write down your name, student id and email.
\footnotesize{\color{blue}$*$ Name:Zehao Wang\quad Student ID:518021910976}
\end{center}

\begin{enumerate}
	\item 
	    Ten people numbered 1 to 10 are lined up in a circle as in the Josephus problem, and every $m$-th person is executed.(The value may be much larger than 10.) Prove that the first three people to go cannot be $10$,$k$, and $k+1$(in this order),for any $k$.
	    \begin{solution}
	    	The first people to go being $10$ means $m\ mod\ 10=0$.This means $m$ is an even number.
	    	The second people to go being $k$ means $m\ mod\ 9=k$.
	    	The third people to go being $k+1$ means $m\ mod\ 8=1$.This means $m$ is an odd number.
	    	
	        However, an integer can not be an even number and an odd number at the same time.
	        Therefore the first three people to go cannot be $10$, $k$, and $k+1$.
	    \end{solution}
	\item 
	    The residue number system($x\ mod\ 3,x\ mod\ 5$) considered in the next has the curious property that $13$ corresponds to (1,3), which looks almost the same. Explain how to find all instance of such a coincidence, without calculating all fifteen pairs of residues. In other words, find all solutions to the congruences
	    \begin{equation*}
	        10x+y\equiv x\ (mod\ 3),\quad 10x+y\equiv y\ (mod\ 5)
	    \end{equation*}
	    \begin{solution}
	    	Let
	    	\begin{equation*}
	    	    10x+y\equiv 10x+6y(\ mod\ 15)
	    	\end{equation*}
	    	we can get
	    	\begin{equation*}
	    	    y\equiv 6y(\ mod\ 15)
	    	\end{equation*}
	    	Namely
	    	\begin{equation*}
	    	    5y\equiv 0(\ mod\ 15)
	    	\end{equation*}
	    	Because $5\equiv 0(mod\ 5)$, therefore $y\equiv0(mod\ 3)$.
	    	
	    	Therefore $y=0$ or $3$.
	    	Correspondingly, $x=0$ or $1$.
	    \end{solution}
	\item 
	    Show that $(3^{77}-1)/2$ is odd and composite.
	    \begin{solution}
	    	\begin{itemize}
	    		\item
	    		Because $3\equiv -1(mod\ 4)$ and $a\equiv b(mod\ m), c\equiv d(mod\ m)\Rightarrow ab\equiv cd(mod\ m)$,
	    		
	    		Therefore $3^{77}\equiv (-1)^{77}(mod\ 4)$.
	    		
	    		Therefore $3^{77}\equiv -1(mod\ 4)$.
	    		
	    		Therefore $3^{77}\equiv 3(mod\ 4)$.
	    		
	    		So there exist integer $k$, such that $3^{77}=4k+3$.
	    		
	    		Therefore $\frac{3^{77}-1}{2}=\frac{4k+2}{2}=2k+1$, which is an odd number.
	    		\item 
	    		$3^{77}-1={3^{7}}^{11}-1=\frac{3^7-1}{2}\times\sum_{k=0}^{10}{3^7}^{k}$
	    		
	    		$3^7-1$ is an even number, so $\frac{3^7-1}{2}$ is an integer.
	    		
	    		Therefore $\frac{3^{77}-1}{2}$ is divisible by $\frac{3^7-1}{2}$, which means it is a composite.
	    	\end{itemize}
	    	
	    	
	    	
	    \end{solution}
	\item 
	    Compute $\phi(999)$
	    \begin{solution}
	    	$999=37\times3^3$.
	    	
	    	Therefore $\phi(999)=\phi(37)\times\phi(3^3)$.
	    	
	    	$\phi(37)=37-1=36$, $\phi(3^3)=3^3-3^2=18$.
	    	
	    	Therefore $\phi(999)=36\times18=648$.
	    \end{solution}
	\item 
	    Find a function $\sigma(n)$ with the property that
	    \begin{equation*}
	        g(n)=\sum_{0\le k\le n}f(k) \Leftrightarrow f(n)=\sum_{0\le k\le n} \sigma(k)g(n-k)
	    \end{equation*}
	    \begin{solution}
	    	\begin{equation*}
	    	    \sigma(n)=
	    	    \begin{cases}
	    	        1&n=0\\
	    	        -1&n=1\\
	    	        0&other\ cases
	    	    \end{cases}
	    	\end{equation*}
	    	
	    	\textbf{Proof.}
	    	\begin{itemize}
	    		\item 
	    		    When $n=0$,$g(0)=f(0)$.Therefore $f(0)=1\times g(0-0)$. Therefore $\sigma(0)=1$.
	    		\item 
	    		    When $n=1$,$g(1)=f(0)+f(1)$.Therefore $f(1)=g(1)-f(0)=g(1)-g(0)$.Therefore $\sigma(1)=-1$.
	    		\item 
	    		    When $n=2$,$g(2)=f(0)+f(1)+f(2)$.By trying to use $g(n)$'s to represent $f(2)$,we can get $\sigma(2)=0$.
	    		\item 
	    		    Assumes that when $n=k_0$($k_0>2$), $\sigma(n)=0$.
	    		    
	    		    Then we have $g(k_0)=\sum_{0\le k\le k_0}f(k)\Leftrightarrow f(k_0)=\sum_{0 \le k\le n}\sigma(k)g(n-k)=g(k_0)-g(k_0-1)$.
	    		    
	    		    Then when $n=k_0+1$, we have $g(k_0+1)=\sum_{0\le k \le k_0}f(k)+f(k_0+1)$.
	    		    
	    		    Therefore $f(k_0+1)=g(k_0+1)-g(k_0)$, which means $\sigma(k_0+1)=0$.
	    		    
	    		    So by induction we can prove when $n\ge 2$, $\sigma(n)=0$.
	    	\end{itemize}
	    	      
	    	      Therefore, the conclusion above is true.
        \end{solution}
	\item 
	    Simplify the formula $\sum_{d\setminus m}\sum_{k\setminus d}\mu(k)g(d/k)$
	    \begin{solution}
	    	\begin{equation}
	    	    \sum_{d\setminus m}\sum_{k\setminus d}\mu(k)g(d/k)
	    	\end{equation}
	    	\begin{equation*}
	    	    =\sum_{d\setminus m}\sum_{k\setminus d}\mu(d/k)g(k)
	    	\end{equation*}
	    	\begin{equation*}
	    	    =\sum_{k\setminus m}\sum_{d\setminus (m/k)} \mu(d)g(k)
	    	\end{equation*}
	    	\begin{equation*}
	    	    =\sum_{k\setminus m}g(k)\times[m/k=1]
	    	\end{equation*}
	    	\begin{equation*}
	    	    =g(m)
	    	\end{equation*}
	    \end{solution}
	\item 
	    A positive integer $n$ is called squarefree if it is not divisible by $m^2$ for any $m>1$. Find a necessary and sufficient condition that $n$ is squarefree,
	    \begin{enumerate}
	    	\item
	    	    in terms of the prime-exponent representation(4.11) of $n$.
	    	\item 
	    	    in terms of $\mu(n)$.
	    \end{enumerate}
        \begin{solution}
        	\begin{enumerate}
        		\item
        		    \begin{equation*}
        		    n=\prod_{p}p^{n_p}.
        		    \end{equation*}
        		    
        		    if $n$ is squarefree, then $n$ should not be divisible by square of its any prime factor.
        		    
        		    To make this come true, its every $n^p$ should be less than $1$.
        		    
        		    Therefore the condition is: For any prime $p$, $n_p\le1$.
        		\item 
        		    According to the definition of $\mu$-function, $\mu(n)=0$ iff $n$ is divisible by a $p^2$.
        		    Therefore $\mu(n)$ should be $0$.
        	\end{enumerate}
        \end{solution}
    \item 
        What is $11^4$? Why is this number easy to compute, for a person who knows binomial coefficients?
        \begin{solution}
        	$11^4=14641$.
        	
        	\begin{equation*}
        	    11^4=(10+1)^4
        	\end{equation*}
            \begin{equation*}    
        	    =\sum_{k} \binom{4}{k} 10^k\times 1^{4-k}
        	\end{equation*}  
            \begin{equation*} 
        	    =\sum_{k=0}^{4} \binom{4}{k} 10^k
        	\end{equation*}
            \begin{equation*}   
        	    =14641
        	\end{equation*}
        \end{solution}
    \item 
        For which value(s) of $k$ is $\binom{n}{k}$ a maximum, when $n$ is a given positive integer? Prove your answer.
        \begin{solution}
        	For any integer $n$, we have:
        	\begin{equation*}
        	    \binom{n}{k}=\frac{n^{\underline{k}}}{k!}
        	\end{equation*}
        	
        	Assume that 
        	\begin{equation}\label{T2_1}
        	   \frac {\binom{n}{k}}{\binom{n}{k-1}}=\frac{n-k+1}{k}\ge 1
        	\end{equation}
        	
        	and
        	
        	\begin{equation}\label{T2_2}
        	   \frac{\binom{n}{k}}{\binom{n}{k+1}}=\frac{k+1}{n-k}\ge 1
        	\end{equation}
        	
        	According to (\ref{T2_1}), we can get $k\le \frac{n+1}{2}$.
        	
        	According to (\ref{T2_2}), we can get $k \ge \frac{n-1}{2}$.
        	
        	Therefore,
        	
        	When $n$ is an odd number and $k=\frac{n+1}{2}$ or $\frac{n-1}{2}$, $\binom{n}{k}$ is a maximum.
        	
        	When $n$ is an even number and $k=\frac{n}{2}$, $\binom{n}{k}$ is a maximum.
        	
        \end{solution}
    \item 
        Prove the hexagon property,
        \begin{equation*}
            \binom{n-1}{k-1}\binom{n}{k+1}\binom{n+1}{k}=\binom{n-1}{k}\binom{n+1}{k+1}\binom{n}{k-1}
        \end{equation*}
        
        \begin{proof}
        	\begin{equation*}
        	    \binom{n-1}{k-1}\binom{n}{k+1}\binom{n+1}{k}
        	\end{equation*}
        	\begin{equation*}
        	    =\frac{(n-1)^{\underline{k-1}}}{(k-1)!}\times\frac{n^{\underline{k+1}}}{(k+1)!}\times\frac{(n+1)^{\underline{k}}}{k!}
        	\end{equation*}
        	\begin{equation*}
        	    =\frac{(n-1)(n-2)\cdots(n-k+1)}{(k-1)!}\times\frac{n(n-1)(n-2)\cdots(n-k)}{(k+1)!}\times\frac{(n+1)n\cdots(n-k+2)}{k!}
        	\end{equation*}
        	\begin{equation*}
        	    =\frac{n(n-1)(n-2)\cdots(n-k)}{(k-1)!}\times\frac{(n-1)(n-2)\cdots(n-k)}{k!}\times\frac{(n+1)n(n-1)\cdots(n-k+2)(n-k+1)}{(k+1)!}
        	\end{equation*}
        	\begin{equation*}
        	    =\frac{n^{\underline{k-1}}}{(k-1)!}\times\frac{(n-1)^{\underline{k}}}{k!}\times\frac{(n+1)^{\underline{k+1}}}{(k+1)!}
        	\end{equation*}
        	\begin{equation*}
        	    =\binom{n-1}{k}\binom{n+1}{k+1}\binom{n}{k-1}
        	\end{equation*}
        	
        \end{proof}
    \item 
        Evaluate $\binom{-1}{k}$ by negating (actually un-negating) its upper index.
        \begin{solution}
        	We have 
        	\begin{equation*}
        	    \binom{n}{k}=(-1)^k\binom{n-k-1}{k!}
        	\end{equation*}
        	
        	Therefore
        	\begin{equation*}
        	    \binom{-1}{k}=(-1)^k\binom{k-(-1)-1}{k}=(-1)^k\binom{k}{k}=(-1)^k[k\ge 0].
        	\end{equation*}
        \end{solution}
    \item 
        Let $p$ be prime. Show that $\binom{p}{k}\ mod\ p=0$ for $0<k<p$. What does this imply about the binomial coefficients $\binom{p-1}{k}$?
        \begin{solution}
        	Because
        	
        	\begin{equation*}
        	\binom{p}{k}=\frac{p^{\underline{k}}}{k!}=p\times\frac{(p-1)^{\underline{k-1}}}{k!}        	
        	\end{equation*}
        	
        Therefore $\binom{p}{k}$ has a factorial $p$. So it is divisible by $p$, namely $\binom{p}{k}\ mod\ p=0$.
        
        Because we have
        \begin{equation*}
            \binom{p}{k}=\binom{p-1}{k-1}+\binom{p-1}{k}
        \end{equation*}
        
        and
        \begin{equation*}
            \binom{p}{k}\equiv 0(mod\ p)
        \end{equation*}
        
        Therefore
        \begin{equation}\label{EQ}
            \binom{p-1}{k-1}+\binom{p-1}{k}\equiv 0(mod\ p)
        \end{equation}
        
        As for $\binom{p-1}{k-1}$,we have
        \begin{equation*}
            \binom{p-1}{k-1}=\frac{(p-1)\cdots(p-k+1)}{(k-1)!}\equiv\frac{(-1)(-2)\cdots(-k+1)}{(k-1)!}(mod\ p)
        \end{equation*}
        Because
        \begin{equation*}
            \frac{(-1)(-2)\cdots(-k+1)}{(k-1)!}=(-1)^{k-1}
        \end{equation*}
        
        Therefore $\binom{p-1}{k-1}\equiv (-1)^{k-1}(\mod p)$.
        
        According to (\ref{EQ}), we can know $\binom{p}{k-1}\equiv (-1)^k(mod\ p)$.
    \end{solution}
    \item 
        Fix up the text's derivation in Problem 6, Section 5.2, by correctly applying symmetry.
        \begin{solution}
        	 \begin{equation*}
        	     \frac{1}{n+1} \sum_{k} \binom{n+k}{k} \binom{n+1}{k+1}(-1)^k
        	 \end{equation*}
        	 \begin{equation*}
        	     =\frac{1}{n+1} \sum_{k\ge 0} \binom{n+k}{n} \binom{n+1}{k+1}(-1)^k
        	 \end{equation*}
        	 \begin{equation*}
        	     =\frac{1}{n+1} \sum_{k} \binom{n+k}{n} \binom{n+1}{k+1}(-1)^k
        	 \end{equation*}
        	 \begin{equation*}
        	     =-\frac{1}{n+1} \binom{n-1}{n}\binom{n+1}{0}(-1)^{-1}
        	 \end{equation*}
        \end{solution}
\end{enumerate}

%========================================================================
\end{document}
