\documentclass[12pt,a4paper]{article}
\usepackage{ctex}
\usepackage{amsmath,amscd,amsbsy,amssymb,latexsym,url,bm,amsthm}
\usepackage{epsfig,graphicx,subfigure}
\usepackage{enumitem,balance}
\usepackage{wrapfig}
\usepackage{mathrsfs,euscript}
\usepackage[usenames]{xcolor}
\usepackage{hyperref}
\usepackage[vlined,ruled,linesnumbered]{algorithm2e}
\hypersetup{colorlinks=true,linkcolor=black}

\newtheorem{theorem}{Theorem}
\newtheorem{lemma}[theorem]{Lemma}
\newtheorem{proposition}[theorem]{Proposition}
\newtheorem{corollary}[theorem]{Corollary}
\newtheorem{exercise}{Exercise}
\newtheorem*{solution}{Solution}
\newtheorem{definition}{Definition}
\theoremstyle{definition}

\renewcommand{\thefootnote}{\fnsymbol{footnote}}

\newcommand{\postscript}[2]
 {\setlength{\epsfxsize}{#2\hsize}
  \centerline{\epsfbox{#1}}}

\renewcommand{\baselinestretch}{1.0}

\setlength{\oddsidemargin}{-0.365in}
\setlength{\evensidemargin}{-0.365in}
\setlength{\topmargin}{-0.3in}
\setlength{\headheight}{0in}
\setlength{\headsep}{0in}
\setlength{\textheight}{10.1in}
\setlength{\textwidth}{7in}
\makeatletter \renewenvironment{proof}[1][Proof] {\par\pushQED{\qed}\normalfont\topsep6\p@\@plus6\p@\relax\trivlist\item[\hskip\labelsep\bfseries#1\@addpunct{.}]\ignorespaces}{\popQED\endtrivlist\@endpefalse} \makeatother
\makeatletter
\renewenvironment{solution}[1][Solution] {\par\pushQED{\qed}\normalfont\topsep6\p@\@plus6\p@\relax\trivlist\item[\hskip\labelsep\bfseries#1\@addpunct{.}]\ignorespaces}{\popQED\endtrivlist\@endpefalse} \makeatother

\begin{document}
\noindent

%========================================================================
\noindent\framebox[\linewidth]{\shortstack[c]{
\Large{\textbf{CS499 Homework 2}}\vspace{1mm}}}
\begin{center}
% Please write down your name, student id and email.
\footnotesize{\color{blue}$*$ Name:Zehao Wang\quad Student ID:518021910976}
\end{center}

\begin{enumerate}
    \item 
    What does notation $\sum_{k=4}^{0}q_k$ mean?
    \begin{solution}
        There are 3 ways to define $\sum_{k=4}^{0}q_k$:
            \begin{enumerate}
            	\item 
            	    $\sum_{k=4}^{0}q_k$=$\sum_{k=0}^{4}q_k=q_0+q_1+q_2+q_4$.
            	\item 
            	    $\sum_{k=4}^{0}q_k=0$.
            	\item 
            	    $\sum_{k=4}^{0}q_k=-q_0-q_1-q_2-q_3-q_4$.
            \end{enumerate}
        
        Considering that $\sum_{k\le m}q_k-\sum_{k\le n}q_k=\sum_{k=n}^{m}q_k$, $\sum_{k=4}^{0}q_k$ had better be defined as $\sum_{q=4}^{0}=\sum_{k\le0}q_k-\sum_{k\le4}q_k=-q_0-q_1-q_2-q_3-q_4$.
    \end{solution}
    \item
    	Simplify the expression $x\cdot([x>0]-[x<0])$.
    	\begin{solution}
    		\begin{enumerate}
    			\item
            		When $x>0$,$[x>0]=1$,$[x<0]=0$.So $x\cdot([x>0]-[x<0])=x\cdot(1-0)=x$.
        		\item
            		When $x=0$,$[x>0]=[x<0]=0$.So $x\cdot([x>0]-[x<0])=0=x$.
        		\item
            		When $x<0$,$[x>0]=0$,$[x<0]=1$.So $x\cdot([x>0]-[x<0])=-x$.
    		\end{enumerate}
    	Therefore
    	\begin{equation*}
        y=
        \begin{cases}
            x & x\ge0\\
            -x & x<0
        \end{cases}
		\end{equation*}
    	\end{solution}
    \item 
        Demonstrate your understanding of $\Sigma$-notation by writing out the sums $\sum_{0\le k\le5}a_k$ and $\sum_{0\le k^2\le5}a_{k^2}$ in full.
        \begin{solution}
        	
        	$\sum_{0\le k\le 5} a_k=a_0+a_1+a_2+a_3+a_4+a_5$.
        	
        	$\sum_{0\le k^2\le5}a_{k^2}=a_{0^2}+a_{1^2}+a_{(-1)^2}+a_{2^2}+a_{(-2)^2}=a_0+2a_1+2a_4$.
        	
        \end{solution}
    \item 
        Express the triple sum $\sum_{1 \le i < j < k \le 4}a_{ijk}$ as a three-fold summation(with three $\sum$'s),
        \begin{enumerate}
        	\item
        	    summing first on $k$, then $j$, then $i$;
        	\item 
        	    summing first on $i$, then $j$, then $k$.
        \end{enumerate}
        Also write your triple sums out in full without the $\Sigma$-notation, using parentheses to show what is being added together first.
        \begin{solution}
        	\quad\\
        	\begin{enumerate}
        		\item 
        		    $\sum_{i=1}^{4}\sum_{j=i+1}^{4}\sum_{k=j+1}^{4}a_{ijk}=\sum_{i=1}^{2}\sum_{j=i+1}^{3}\sum_{k=j+1}^4a_{ijk}=(a_{123}+a_{124})+(a_{134})+(a_{234})$.
        		\item 
        		    $\sum_{k=1}^4\sum_{j=1}^{k-1}\sum_{i=1}^{j-1}a_{ijk}=\sum_{k=3}^4\sum_{j=2}^{k-1}\sum_{i=1}^{j-1}a_ijk=(a_{123})+((a_{124})+(a_{134}+a_{234}))$
        	\end{enumerate}
        \end{solution}
    \item 
        What's wrong with the following derivation?
        
        \begin{equation*}
            (\sum_{j=1}^{n}a_j)(\sum_{k=1}^{n}\frac{1}{a_k})=\sum_{j=1}^{n}\sum_{k=1}^{n}\frac{a_j}{a_k}
            =\sum_{j=1}^{n}\sum_{k=1}^{n}\frac{a_k}{a_k}=\sum_{k=1}^{n}n=n^2
        \end{equation*}
        \begin{solution}
        	The $j$ in the second step should not be changed into $k$. $j$ and $k$ are irrelevant variables.But the change turns $j$ into a relevant variable to $k$.This changes the value of elements to be calculated.
        \end{solution}
    \item 
        What is the value of $\sum_{k}[1 \le j \le k \le n]$, as a function of $j$ and $n$?
        \begin{solution}
        	\begin{equation*}
        	    y=
        	    \begin{cases}
        	        n-j+1 & (1\le j \le n)\\
        	        0 & (other\quad cases)
        	    \end{cases}
        	\end{equation*}
        	For any $k\in \mathbb{N}$, $[1 \le j \le k \le n]=1$ iff $1 \le j \le n$ and $j \le k \le n$.
            If $j$,$n$ satisfies that $1 \le j \le n$, then there are some elements between $j$ and $n$.Therefore the value should be $n-j+1$.
            If $j$,$n$ does not satisfy that $1 \le j \le n$, it should be $0$.
        \end{solution}
    \item 
        Let $\nabla f(x)=f(x)-f(x-1)$.What is $\nabla(x^{\overline{m}})$?
        \begin{solution}
        	$x^{\overline{m}}=x(x+1)(x+2)\cdots(x+m-1)$.
        	
        	Therefore $f(x)=x(x+1)(x+2)\cdots(x+m-1)$.
        	
            $f(x-1)=(x-1)x(x+1)(x+2)\cdots(x+m-2)$.
        	
        	Therefore $\nabla f(x)=f(x)-f(x-1)=[x(x+1)(x+2)\cdots(x+m-2)][(x+m-1)-(x-1)]$
        	
        	$=mx^{\overline{m-1}}$.
        \end{solution}
    \item 
        What is the value of $0^{\underline{m}}$, when $m$ is a given interger?
        \begin{solution}
        	If $m>0$, then $0^{\underline{m}}=0\times1\times2\times\cdots(m-1)=0$.\\
        	If $m<0$, then $0^{\underline{m}}=\frac{1}{(0+1)(0+2)\cdots(0+m)}=\frac{1}{|m|!}$.\\
        	If $m=0$, then $0^{\underline{m}}=0^{\underline{0}}=1$.\\
        	To sum up,
        	\begin{equation*}
        	    0^{\underline{m}}=
        	        \begin{cases}
        	            0&m> 0\\
        	            \frac{1}{|m|!}&m\le 0
        	        \end{cases}
        	\end{equation*}
        	
        \end{solution}
    \item 
        What is the law of exponents for rising factorial powers, analogous to (2.52)?Use this to define $x^{\overline{-n}}$.
        \begin{solution}
        	$x^{\overline{m+n}}=x^{\overline{m}}(x+m)^{\overline{n}}$.
        	
        	\quad$x^{\overline{0}}$
        	
        	$=x^{\overline{(-n)+n}}$
        	
        	$=x^{\overline{-n}} (x-n)^{\overline{n}}$
        	
        	$=x^{\overline{-n}}[(x-n)(x-n+1)\cdots(x-1)]$
        	
        	$=1$.
        	
        	Therefore $x^{\overline{-n}}=\frac{1}{(x-1)^{\underline{n}}}$.
        \end{solution}
    \item 
        The text derives the following formula for the difference of a product:
        \begin{equation*}
            \Delta(uv)=u\Delta v+Ev\Delta u
        \end{equation*}
        How can this formula be correct, when the left-hand side is symmetric with respect to $u$ and $v$ but the right-hand side is not?
        \begin{solution}
        	\quad\\
        	$u\Delta v+Ev\Delta u=u(x)(v(x+1)-v(x))+v(x+1)(u(x+1)-u(x))=uEv-uv+EuEv-uEv=u(x+1)v(x+1)-u(x)v(x)=\Delta(uv)$.
        	
        	$v\Delta u+Eu\Delta v=v(x)(u(x+1)-u(x))+u(x+1)(v(x+1)-v(x))=vEu-uv+EuEv-vEu=\Delta(uv) $.
        	
        	Therefore it does not matter the right-hand side is not symmetric.They are the same.
        \end{solution}
        
\end{enumerate}

%========================================================================
\end{document}
